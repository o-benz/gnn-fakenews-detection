\section{Introduction}
% Brief overview of fake news detection
% Summary of past work (cite 2-3 papers briefly)
% Why GNNs? Transition into theoretical foundations

Fake news, often defined as low-quality or fabricated news intended to mislead readers, has emerged as a significant societal problem in the age of social media~\cite{Shu2017}. Online platforms enable the rapid and wide dissemination of misinformation, allowing fake news to reach millions of users with ease. The extensive spread of fake news can distort public opinion and undermine trust in institutions, leading to harmful real-world consequences~\cite{Shu2017}. For example, empirical studies have shown that false news propagates faster and farther than true news in social networks, amplifying its societal impact~\cite{Vosoughi2018}. These concerns have driven an urgent demand for effective fake news detection methods.

Detecting fake news, however, is a non-trivial task due to several inherent challenges. Unlike generic spam or factual errors, fake news is usually {\it intentionally} written to appear credible, making it difficult to identify based on content alone. Traditional approaches that rely purely on text analysis or metadata often fall short because malicious actors carefully craft fake stories to mimic the style of legitimate news~\cite{Shu2017}. As a result, recent research highlights the importance of incorporating auxiliary information beyond the news content itself. In particular, the way news spreads through social media—such as user engagements, comments, and sharing patterns—provides crucial contextual signals for detection~\cite{Shu2017}. Integrating this social context is essential: for instance, users tend to spread misinformation that aligns with their pre-existing beliefs (a form of confirmation bias), and analyzing who is sharing an article and how it diffuses can help discern fake news from real news. Yet, exploiting such information is complex, as social media data is often massive, noisy, and heterogeneous, combining text, user attributes, and network structure~\cite{Shu2017}. This complexity necessitates advanced machine learning techniques capable of fusing content and context.

Graph-based learning has recently gained traction as a promising direction to address these challenges in fake news detection. Social networks naturally form graph-structured data (with users, posts, and topics as nodes and various relations as edges), and fake news propagation can be viewed as a diffusion process on a graph. By modeling news dissemination as a graph problem, one can capture relational patterns (e.g., which users or communities are interconnected and prone to share the same misinformation) that are invisible to text-only methods. **Graph Neural Networks (GNNs)**, a class of deep learning models designed for graph-structured data, are particularly well-suited for this task~\cite{SanchezLengeling2021}. GNNs operate by iteratively aggregating information from a node’s neighbors in the network, effectively learning representations that encode both the attributes of the node (such as the content of a news article) and the structure of its local neighborhood (such as the users who spread that article and their connections)~\cite{SanchezLengeling2021}. This capability allows GNN-based approaches to combine textual features with social context, aligning well with the need to use auxiliary network information in fake news detection. Indeed, GNN models have started to see practical applications in domains like fraud and misinformation detection, where relational data is key to performance~\cite{SanchezLengeling2021}.

Several recent works demonstrate the effectiveness of GNNs for fake news identification. Notably, Zhang \emph{et al.}~\cite{Zhang2024} propose \textit{GNN-FakeNews}, a graph-based fake news detection framework that integrates social network cues with content features. In their approach, each news piece is represented as a node in a large heterogeneous graph, connected with user nodes that have engaged with the news (e.g., by sharing or commenting) as well as with other news nodes via shared user engagements. By applying a GNN over this news-user interaction graph, the model can learn high-level representations of news articles that reflect both what the article contains and how it spreads through the network. This framework, which jointly models news content and propagation structure, has achieved state-of-the-art performance on benchmark datasets for fake news detection~\cite{Zhang2024}. The authors report that incorporating graph connectivity—such as user preference patterns and propagation pathways—substantially improves detection accuracy compared to traditional text-only classifiers. These results echo the broader trend in the literature: leveraging social context via graph neural networks leads to more robust fake news detection systems.

Motivated by the success of graph-based methods, our work explores Graph Neural Networks as a powerful approach to fake news detection. In the following, we transition to the theoretical underpinnings of GNN models, explaining how they operate and why they are well-suited for this task. This foundation will inform the design of our fake news detection methodology presented later in the report.
